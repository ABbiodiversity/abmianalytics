\documentclass[]{article}
\usepackage{lmodern}
\usepackage{amssymb,amsmath}
\usepackage{ifxetex,ifluatex}
\usepackage{fixltx2e} % provides \textsubscript
\ifnum 0\ifxetex 1\fi\ifluatex 1\fi=0 % if pdftex
  \usepackage[T1]{fontenc}
  \usepackage[utf8]{inputenc}
\else % if luatex or xelatex
  \ifxetex
    \usepackage{mathspec}
  \else
    \usepackage{fontspec}
  \fi
  \defaultfontfeatures{Ligatures=TeX,Scale=MatchLowercase}
\fi
% use upquote if available, for straight quotes in verbatim environments
\IfFileExists{upquote.sty}{\usepackage{upquote}}{}
% use microtype if available
\IfFileExists{microtype.sty}{%
\usepackage{microtype}
\UseMicrotypeSet[protrusion]{basicmath} % disable protrusion for tt fonts
}{}
\usepackage[margin=1in]{geometry}
\usepackage{hyperref}
\hypersetup{unicode=true,
            pdftitle={Modeling linear feature effects},
            pdfauthor={P. Solymos},
            pdfborder={0 0 0},
            breaklinks=true}
\urlstyle{same}  % don't use monospace font for urls
\usepackage{color}
\usepackage{fancyvrb}
\newcommand{\VerbBar}{|}
\newcommand{\VERB}{\Verb[commandchars=\\\{\}]}
\DefineVerbatimEnvironment{Highlighting}{Verbatim}{commandchars=\\\{\}}
% Add ',fontsize=\small' for more characters per line
\usepackage{framed}
\definecolor{shadecolor}{RGB}{248,248,248}
\newenvironment{Shaded}{\begin{snugshade}}{\end{snugshade}}
\newcommand{\KeywordTok}[1]{\textcolor[rgb]{0.13,0.29,0.53}{\textbf{#1}}}
\newcommand{\DataTypeTok}[1]{\textcolor[rgb]{0.13,0.29,0.53}{#1}}
\newcommand{\DecValTok}[1]{\textcolor[rgb]{0.00,0.00,0.81}{#1}}
\newcommand{\BaseNTok}[1]{\textcolor[rgb]{0.00,0.00,0.81}{#1}}
\newcommand{\FloatTok}[1]{\textcolor[rgb]{0.00,0.00,0.81}{#1}}
\newcommand{\ConstantTok}[1]{\textcolor[rgb]{0.00,0.00,0.00}{#1}}
\newcommand{\CharTok}[1]{\textcolor[rgb]{0.31,0.60,0.02}{#1}}
\newcommand{\SpecialCharTok}[1]{\textcolor[rgb]{0.00,0.00,0.00}{#1}}
\newcommand{\StringTok}[1]{\textcolor[rgb]{0.31,0.60,0.02}{#1}}
\newcommand{\VerbatimStringTok}[1]{\textcolor[rgb]{0.31,0.60,0.02}{#1}}
\newcommand{\SpecialStringTok}[1]{\textcolor[rgb]{0.31,0.60,0.02}{#1}}
\newcommand{\ImportTok}[1]{#1}
\newcommand{\CommentTok}[1]{\textcolor[rgb]{0.56,0.35,0.01}{\textit{#1}}}
\newcommand{\DocumentationTok}[1]{\textcolor[rgb]{0.56,0.35,0.01}{\textbf{\textit{#1}}}}
\newcommand{\AnnotationTok}[1]{\textcolor[rgb]{0.56,0.35,0.01}{\textbf{\textit{#1}}}}
\newcommand{\CommentVarTok}[1]{\textcolor[rgb]{0.56,0.35,0.01}{\textbf{\textit{#1}}}}
\newcommand{\OtherTok}[1]{\textcolor[rgb]{0.56,0.35,0.01}{#1}}
\newcommand{\FunctionTok}[1]{\textcolor[rgb]{0.00,0.00,0.00}{#1}}
\newcommand{\VariableTok}[1]{\textcolor[rgb]{0.00,0.00,0.00}{#1}}
\newcommand{\ControlFlowTok}[1]{\textcolor[rgb]{0.13,0.29,0.53}{\textbf{#1}}}
\newcommand{\OperatorTok}[1]{\textcolor[rgb]{0.81,0.36,0.00}{\textbf{#1}}}
\newcommand{\BuiltInTok}[1]{#1}
\newcommand{\ExtensionTok}[1]{#1}
\newcommand{\PreprocessorTok}[1]{\textcolor[rgb]{0.56,0.35,0.01}{\textit{#1}}}
\newcommand{\AttributeTok}[1]{\textcolor[rgb]{0.77,0.63,0.00}{#1}}
\newcommand{\RegionMarkerTok}[1]{#1}
\newcommand{\InformationTok}[1]{\textcolor[rgb]{0.56,0.35,0.01}{\textbf{\textit{#1}}}}
\newcommand{\WarningTok}[1]{\textcolor[rgb]{0.56,0.35,0.01}{\textbf{\textit{#1}}}}
\newcommand{\AlertTok}[1]{\textcolor[rgb]{0.94,0.16,0.16}{#1}}
\newcommand{\ErrorTok}[1]{\textcolor[rgb]{0.64,0.00,0.00}{\textbf{#1}}}
\newcommand{\NormalTok}[1]{#1}
\usepackage{graphicx,grffile}
\makeatletter
\def\maxwidth{\ifdim\Gin@nat@width>\linewidth\linewidth\else\Gin@nat@width\fi}
\def\maxheight{\ifdim\Gin@nat@height>\textheight\textheight\else\Gin@nat@height\fi}
\makeatother
% Scale images if necessary, so that they will not overflow the page
% margins by default, and it is still possible to overwrite the defaults
% using explicit options in \includegraphics[width, height, ...]{}
\setkeys{Gin}{width=\maxwidth,height=\maxheight,keepaspectratio}
\IfFileExists{parskip.sty}{%
\usepackage{parskip}
}{% else
\setlength{\parindent}{0pt}
\setlength{\parskip}{6pt plus 2pt minus 1pt}
}
\setlength{\emergencystretch}{3em}  % prevent overfull lines
\providecommand{\tightlist}{%
  \setlength{\itemsep}{0pt}\setlength{\parskip}{0pt}}
\setcounter{secnumdepth}{0}
% Redefines (sub)paragraphs to behave more like sections
\ifx\paragraph\undefined\else
\let\oldparagraph\paragraph
\renewcommand{\paragraph}[1]{\oldparagraph{#1}\mbox{}}
\fi
\ifx\subparagraph\undefined\else
\let\oldsubparagraph\subparagraph
\renewcommand{\subparagraph}[1]{\oldsubparagraph{#1}\mbox{}}
\fi

%%% Use protect on footnotes to avoid problems with footnotes in titles
\let\rmarkdownfootnote\footnote%
\def\footnote{\protect\rmarkdownfootnote}

%%% Change title format to be more compact
\usepackage{titling}

% Create subtitle command for use in maketitle
\newcommand{\subtitle}[1]{
  \posttitle{
    \begin{center}\large#1\end{center}
    }
}

\setlength{\droptitle}{-2em}
  \title{Modeling linear feature effects}
  \pretitle{\vspace{\droptitle}\centering\huge}
  \posttitle{\par}
  \author{P. Solymos}
  \preauthor{\centering\large\emph}
  \postauthor{\par}
  \predate{\centering\large\emph}
  \postdate{\par}
  \date{2015-07-15}


\begin{document}
\maketitle

\section{The model}\label{the-model}

\(N = A_h D_h + A_l D_l\) where \(h\) is surrounding habitat in a
circular buffer and \(l\) is the linear feature, \(A\) is area, \(D\) is
density, \(N\) is abundance at a point count.

\(N = (1-p) A D_h + p A \delta D_h\), where \(A=A_h + A_l\) and
\(D_l = \delta D_h\). It follows that
\(D = N/A = D_h (1 - p + p \delta)\).

\section{The problem}\label{the-problem}

We model \(D_h = exp(X \beta)\), and
\(D = exp(X \beta) exp(\beta_l p)\). But the scaling with \(p\) is not
the same in the intuitive model above, and the model that is easy to
implement in a \texttt{glm} framework. Let us plot \(D/D_h\) in 2
different ways:

\begin{Shaded}
\begin{Highlighting}[]
\NormalTok{delta <-}\StringTok{ }\FloatTok{0.1}
\NormalTok{p <-}\StringTok{ }\KeywordTok{seq}\NormalTok{(}\DecValTok{0}\NormalTok{, }\DecValTok{1}\NormalTok{, }\DataTypeTok{by =} \FloatTok{0.01}\NormalTok{)}
\NormalTok{op <-}\StringTok{ }\KeywordTok{par}\NormalTok{(}\DataTypeTok{mfrow=}\KeywordTok{c}\NormalTok{(}\DecValTok{1}\NormalTok{,}\DecValTok{3}\NormalTok{))}
\KeywordTok{plot}\NormalTok{(p, }\KeywordTok{exp}\NormalTok{(}\KeywordTok{log}\NormalTok{(delta)}\OperatorTok{*}\NormalTok{p), }\DataTypeTok{type =} \StringTok{"l"}\NormalTok{, }\DataTypeTok{col =} \DecValTok{2}\NormalTok{, }\DataTypeTok{lwd =} \DecValTok{2}\NormalTok{)}
\KeywordTok{plot}\NormalTok{(p, }\DecValTok{1}\OperatorTok{-}\NormalTok{p}\OperatorTok{+}\NormalTok{p}\OperatorTok{*}\NormalTok{delta, }\DataTypeTok{type =} \StringTok{"l"}\NormalTok{, }\DataTypeTok{col =} \DecValTok{2}\NormalTok{, }\DataTypeTok{lwd =} \DecValTok{2}\NormalTok{)}
\KeywordTok{plot}\NormalTok{(}\KeywordTok{exp}\NormalTok{(}\KeywordTok{log}\NormalTok{(delta)}\OperatorTok{*}\NormalTok{p), }\DecValTok{1}\OperatorTok{-}\NormalTok{p}\OperatorTok{+}\NormalTok{p}\OperatorTok{*}\NormalTok{delta, }\DataTypeTok{type =} \StringTok{"l"}\NormalTok{, }\DataTypeTok{col =} \DecValTok{2}\NormalTok{, }\DataTypeTok{lwd =} \DecValTok{2}\NormalTok{)}
\KeywordTok{abline}\NormalTok{(}\DecValTok{0}\NormalTok{, }\DecValTok{1}\NormalTok{, }\DataTypeTok{lty =} \DecValTok{2}\NormalTok{)}
\end{Highlighting}
\end{Shaded}

\includegraphics{modeling-linear-features_files/figure-latex/unnamed-chunk-1-1.pdf}

\begin{Shaded}
\begin{Highlighting}[]
\KeywordTok{par}\NormalTok{(op)}
\end{Highlighting}
\end{Shaded}

\section{The solution}\label{the-solution}

What would be a function of \(p\) that would best approximate the
intuitive model?

We will compare polynomial terms to estimate bias when the data is
simulated under the intuitive model.

\begin{Shaded}
\begin{Highlighting}[]
\KeywordTok{set.seed}\NormalTok{(}\DecValTok{1234}\NormalTok{)}
\NormalTok{n <-}\StringTok{ }\DecValTok{1000}
\NormalTok{Dh <-}\StringTok{ }\DecValTok{2}
\NormalTok{delta <-}\StringTok{ }\FloatTok{0.1}
\NormalTok{p <-}\StringTok{ }\KeywordTok{runif}\NormalTok{(n, }\DecValTok{0}\NormalTok{, }\DecValTok{1}\NormalTok{)}
\NormalTok{p <-}\StringTok{ }\NormalTok{p[}\KeywordTok{order}\NormalTok{(p)]}
\NormalTok{lam <-}\StringTok{ }\NormalTok{(}\DecValTok{1}\OperatorTok{-}\NormalTok{p)}\OperatorTok{*}\NormalTok{Dh }\OperatorTok{+}\StringTok{ }\NormalTok{p}\OperatorTok{*}\NormalTok{delta}\OperatorTok{*}\NormalTok{Dh}
\NormalTok{y <-}\StringTok{ }\KeywordTok{rpois}\NormalTok{(n, lam)}

\KeywordTok{summary}\NormalTok{(m <-}\StringTok{ }\KeywordTok{glm}\NormalTok{(y }\OperatorTok{~}\StringTok{ }\NormalTok{p, }\DataTypeTok{family=}\NormalTok{poisson))}
\end{Highlighting}
\end{Shaded}

\begin{verbatim}
## 
## Call:
## glm(formula = y ~ p, family = poisson)
## 
## Deviance Residuals: 
##     Min       1Q   Median       3Q      Max  
## -2.1053  -1.0344  -0.2331   0.5413   3.0909  
## 
## Coefficients:
##             Estimate Std. Error z value Pr(>|z|)    
## (Intercept)  0.79732    0.05227   15.25   <2e-16 ***
## p           -1.75762    0.11497  -15.29   <2e-16 ***
## ---
## Signif. codes:  0 '***' 0.001 '**' 0.01 '*' 0.05 '.' 0.1 ' ' 1
## 
## (Dispersion parameter for poisson family taken to be 1)
## 
##     Null deviance: 1288.8  on 999  degrees of freedom
## Residual deviance: 1037.3  on 998  degrees of freedom
## AIC: 2488.1
## 
## Number of Fisher Scoring iterations: 5
\end{verbatim}

\begin{Shaded}
\begin{Highlighting}[]
\KeywordTok{exp}\NormalTok{(}\KeywordTok{coef}\NormalTok{(m)[}\DecValTok{1}\NormalTok{]) }\CommentTok{# Dh}
\end{Highlighting}
\end{Shaded}

\begin{verbatim}
## (Intercept) 
##    2.219595
\end{verbatim}

\begin{Shaded}
\begin{Highlighting}[]
\KeywordTok{exp}\NormalTok{(}\KeywordTok{coef}\NormalTok{(m)[}\DecValTok{1}\NormalTok{] }\OperatorTok{+}\StringTok{ }\KeywordTok{coef}\NormalTok{(m)[}\DecValTok{2}\NormalTok{]) }\CommentTok{# Dl}
\end{Highlighting}
\end{Shaded}

\begin{verbatim}
## (Intercept) 
##   0.3827797
\end{verbatim}

\begin{Shaded}
\begin{Highlighting}[]
\KeywordTok{exp}\NormalTok{(}\KeywordTok{coef}\NormalTok{(m)[}\DecValTok{2}\NormalTok{]) }\CommentTok{# delta}
\end{Highlighting}
\end{Shaded}

\begin{verbatim}
##         p 
## 0.1724547
\end{verbatim}

\begin{Shaded}
\begin{Highlighting}[]
\NormalTok{f <-}\StringTok{ }\KeywordTok{fitted}\NormalTok{(m)}
\KeywordTok{plot}\NormalTok{(p, lam, }\DataTypeTok{ylim =} \KeywordTok{c}\NormalTok{(}\DecValTok{0}\NormalTok{, }\KeywordTok{max}\NormalTok{(lam, f)))}
\KeywordTok{lines}\NormalTok{(p, f)}
\end{Highlighting}
\end{Shaded}

\includegraphics{modeling-linear-features_files/figure-latex/unnamed-chunk-2-1.pdf}

\begin{Shaded}
\begin{Highlighting}[]
\KeywordTok{summary}\NormalTok{(m <-}\StringTok{ }\KeywordTok{glm}\NormalTok{(y }\OperatorTok{~}\StringTok{ }\KeywordTok{I}\NormalTok{(p}\OperatorTok{^}\DecValTok{2}\NormalTok{), }\DataTypeTok{family=}\NormalTok{poisson))}
\end{Highlighting}
\end{Shaded}

\begin{verbatim}
## 
## Call:
## glm(formula = y ~ I(p^2), family = poisson)
## 
## Deviance Residuals: 
##     Min       1Q   Median       3Q      Max  
## -1.8604  -0.9824  -0.3619   0.5252   2.9183  
## 
## Coefficients:
##             Estimate Std. Error z value Pr(>|z|)    
## (Intercept)  0.54841    0.04107   13.35   <2e-16 ***
## I(p^2)      -1.94874    0.13443  -14.50   <2e-16 ***
## ---
## Signif. codes:  0 '***' 0.001 '**' 0.01 '*' 0.05 '.' 0.1 ' ' 1
## 
## (Dispersion parameter for poisson family taken to be 1)
## 
##     Null deviance: 1288.8  on 999  degrees of freedom
## Residual deviance: 1027.8  on 998  degrees of freedom
## AIC: 2478.7
## 
## Number of Fisher Scoring iterations: 5
\end{verbatim}

\begin{Shaded}
\begin{Highlighting}[]
\KeywordTok{exp}\NormalTok{(}\KeywordTok{coef}\NormalTok{(m)[}\DecValTok{1}\NormalTok{]) }\CommentTok{# Dh}
\end{Highlighting}
\end{Shaded}

\begin{verbatim}
## (Intercept) 
##    1.730502
\end{verbatim}

\begin{Shaded}
\begin{Highlighting}[]
\KeywordTok{exp}\NormalTok{(}\KeywordTok{coef}\NormalTok{(m)[}\DecValTok{1}\NormalTok{] }\OperatorTok{+}\StringTok{ }\KeywordTok{coef}\NormalTok{(m)[}\DecValTok{2}\NormalTok{]) }\CommentTok{# Dl}
\end{Highlighting}
\end{Shaded}

\begin{verbatim}
## (Intercept) 
##   0.2465159
\end{verbatim}

\begin{Shaded}
\begin{Highlighting}[]
\KeywordTok{exp}\NormalTok{(}\KeywordTok{coef}\NormalTok{(m)[}\DecValTok{2}\NormalTok{]) }\CommentTok{# delta}
\end{Highlighting}
\end{Shaded}

\begin{verbatim}
##    I(p^2) 
## 0.1424534
\end{verbatim}

\begin{Shaded}
\begin{Highlighting}[]
\NormalTok{f <-}\StringTok{ }\KeywordTok{fitted}\NormalTok{(m)}
\KeywordTok{plot}\NormalTok{(p, lam, }\DataTypeTok{ylim =} \KeywordTok{c}\NormalTok{(}\DecValTok{0}\NormalTok{, }\KeywordTok{max}\NormalTok{(lam, f)))}
\KeywordTok{lines}\NormalTok{(p, f)}
\end{Highlighting}
\end{Shaded}

\includegraphics{modeling-linear-features_files/figure-latex/unnamed-chunk-2-2.pdf}

The linear \texttt{glm} is not a bad approximation.


\end{document}
